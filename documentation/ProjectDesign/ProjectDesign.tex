\documentclass[11pt]{article}
\usepackage{parskip}
\parskip=1\baselineskip \advance\parskip by 0pt plus 2pt
\usepackage[pdftex]{graphicx}
\usepackage[utf8]{inputenc}
\usepackage[margin=1in]{geometry}
\usepackage{url}
\begin{document}

\begin{titlepage}
\begin{center}

\includegraphics[width=0.85\textwidth]{./Logo_Large-cropped_black.png}

\vspace{3 cm}

\textbf{\Huge{Project Design}}

\vspace{1 cm}

\textit{\large{Version 0.1 - 20th November 2011}}

\vspace{4 cm}

\textbf{\Large{Authors:}}

\textbf{Steve Jones} (steve@longwebcam.org)

\end{center}

\end{titlepage}

\setcounter{tocdepth}{2}
\tableofcontents
\clearpage
\pagenumbering{roman}
\section*{Document History}
    \addcontentsline{toc}{section}{Document History}
\begin{table*}[tbhp!]
\begin{tabular}{ c c p{4in} }
\textbf{Version} & \textbf{Date} & \textbf{Notes} \\
0.1 & 20 Nov 2011 & Document development; The version number will be held at 0.1 until all sections have been written. \\
\end{tabular}
\end{table*}


\clearpage
\pagenumbering{arabic}

\section{Introduction}
This document describes the design of the Long Webcam project. It contains detailed descriptions of the actions and use cases the site will support, the database and image storage structures that will support the site's functions, and mock-ups of the major user interface elements of the site. It will also outline the overall Long Webcam system architecture, including the server and database requirements and design.

\section{Cameras}
\label{cameras}
The webcams employed by the project will be maintained by individuals or organisations who will own the cameras together with any equipment required to maintain them and submit images to the Long Webcam site.

\subsection{Types of camera}
\label{sec:camera_types}
The Long Webcam site will support the collection of images from webcams in three different modes:

\begin{itemize}
\item Webcam publishing images to a web page
\item Webcam with software sending images to the Long Webcam site
\item Standalone webcam unit
    \begin{itemize}
    \item With network connection: sends images to the site automatically
    \item Without network connection: images must be manually uploaded to the site
    \end{itemize}
\end{itemize}

Where a webcam is already publishing captured images to a website, adding the daily image to the Long Webcam archive is trivial, and can be handled automatically by the Long Webcam server. Given a fixed URL for the webcam, it can simply download the image at midday (for the time zone in which the camera is located) and add it to the archive.

If a webcam is attached to a standard PC but does not upload captured images to a website, custom software can be written to run on the PC that captures a webcam image and uploads it to the Long Webcam site. As long as the PC is switched on and the software is running, no intervention will be required by the camera owner.

Standalone webcam units are intended for use where standard PC equipment is not appropriate. If they have a connection to the Internet, they will automatically take pictures and upload them to the Long Webcam site as a camera attached to a PC would. If no Internet connection is available, the units will take photos every day and store them on local media. The camera owner will be responsible for periodically collecting the pictures from the webcam unit and uploading them to the Long Webcam site. The exact design and operation of these standalone units is discussed in brief detail in Section~\ref{standalone}.

\subsection{Adding a camera to Long Webcam}
The Long Webcam site's aim is to provide a long-term perspective on the changing world, and its intent is to appeal to a wide range of people. The cameras it uses must therefore record something of interest to that perspective. 'Normal' webcams taking pictures of gardens or other very local arenas are not suitable. Consequently, cameras will only be added to the site after it has been checked by a member of the project team. In order to have a camera considered for inclusion on the Long Webcam site, potential camera hosts must provide certain details to allow the project team to assess its suitability. These details will include:

\begin{itemize}
\item Detail of the camera location and viewing direction, so its position can be checked on, for example, Google Earth.
\item A description of what the camera will be looking at.
\item The potential changes that the camera will observe, and the timescale of those changes.
\item A photograph taken from the camera site to show its field of view.
\end{itemize}

In order to encourage as many submissions as possible, the application process for adding a camera will be very informal. In the first instance, potential hosts can submit as much or as little information as they please - perhaps just the wish to host a camera somewhere. The project team will then establish the necessary requirements through a dialog with the potential host, and assist them to set up the best possible camera for the site's goals.

\subsection{Camera host requirements}
Camera hosts will have to fulfil certain obligations as part of their role in the making the Long Webcam project a success. These obligations will be set out in full to the host before their camera is added to their site, as much to help them decide whether they are able to contribute effectively to the project as to ensure that the project doesn't suffer from poor quality camera locations.

A number of these obligations will be very formal, to ensure that the project can operate as planned and both the project and the camera host can be protected from possible legal issues. These will be presented to the camera host as a Terms and Conditions document to be signed by the host prior to the camera being added to the site. The formal obligations will include, but not be restricted to, the following:

\begin{itemize}
\item The camera must have full legal permission to place the camera in the desired location.
\item The camera host must own full initial copyright on the images from the camera, and therefore be in a position to fulfil the licensing requirements of images for the project.
\item The camera must not be placed in such a way that it captures anything that cannot be legally photographed.
\item The camera must not be placed in such a way that is captures images that are offensive or unsuitable for viewing by all people.
\item The images taken by the camera must not infringe anybody's privacy.
\item The camera host must allow the Long Webcam project to use the captured images for any purpose that helps the project fulfil its aims. This includes disseminating the images to other media outlets.
\item In addition to the special license terms for the Long Webcam project, all images submitted to the Long Webcam project must be release under the Creative Commons Attribution-NonCommercial-ShareAlike 3.0 Unported License (see Appendix A). The Attribution portion of this licence will refer to both the Long Webcam project and the individual camera host. Details of the camera host will always be displayed alongside the images displayed by the project unless the camera host explicitly expresses the desire not to be named in this manner.
\end{itemize}

There is also a list of less formal obligations for camera hosts. These are not legal in nature, but are required to ensure the smooth running of the Long Webcam project.

\begin{itemize}
\item The camera host must understand the long-term nature of the project, and be prepared to maintain the camera for several years at least. The level of maintenance required will depend on the specific location of the camera.
\item The camera host must be prepared to liaise with the Long Webcam project on a regular basis to help resolve maintenance issues (e.g. images not being received) and to provide information required for the project's archives.
\item The camera host must be prepared to help document any changes observed by the camera.
\end{itemize}

\subsection{Standalone webcam units}
\label{standalone}

The Long Webcam project will design standalone webcam units for the placement of cameras where standard equipment is not viable. The units will have various configuration options depending on the availability of mains electricity and Internet access. The fundamental components of the unit will consist of:

\begin{itemize}
\item Camera
\item Control board with software pre-loaded
\item Power supply:
\begin{itemize}
\item Mains power supply unit (if mains electricity is available)
\item Battery pack (if mains is not available)
\end{itemize}
\item Optional network access (wired or wireless, depending on available connections)
\item Local storage for captured images.
\end{itemize}

Each camera will take one image per day, at local midday. If an Internet connection is available, the image will be automatically submitted to the Long Webcam site. (If network issues prevent this, the image will be stored locally until network access is restored.) If no network connection is available, all images will be stored locally. The camera host must retrieve these images periodically and upload them to the site.

Development of the standalone cameras is not an initial goal of the project. These units will be designed more fully at a later date.

\subsection{Using digital cameras}
Many modern digital cameras can be controlled via a PC, so it is not inconceivable that such cameras could be used for the project. Using such cameras will require extra work over the use of simple webcams in terms of supplying power to them and controlling the cameras (switching them on and off, taking pictures and retrieving them from the camera to send to the project's servers). The project should encourage the use of these cameras wherever possible, as they will probably provide much better images than standard webcams. The project may even sourcing old digital cameras and setting up systems for installation.

In the first instance, the support for such cameras will be provided on an \textit{ad hoc} basis.

\clearpage
\section{Stored Data}

The Long Webcam project will store a considerable amount of data in the form of images captured from its cameras. However, the images will be of limited use without context, so a certain amount of additional information will also be stored. This will relate to the cameras themselves, the images captured by the cameras, and other time-specific details of the cameras that will provide historical information regarding the changes seen by each camera over time.

\subsection{Camera Details}

Each camera will be fully described in the project's archive to provide information to users and for the system's operation.

\subsubsection{Technical Information}

The technical specification of the camera itself will be recorded where it is available. At the very least it will include the resolution of the images sent by the camera, which will be readily ascertained from the images themselves. Other camera details may be useful (e.g. for comparing image quality and informing decisions of camera suitability for different locations), and will be recorded for such purpose. The values stored may include:

\begin{itemize}
\item Camera model and make
\item Lens details
\item CCD details
\end{itemize}

Since this is a long-term project, it is likely that cameras will break or grow old and require replacement from time to time. It will be necessary to update the stored camera details as this happens, but the details of previous cameras must be maintained to match the images that were taken before the camera's replacement. The date of each new camera's installation will be sufficient to cross-reference stored images and establish the specific camera with which they were taken.

\subsubsection{Camera Location}
Knowing details of each camera's location is a fundamental requirement of the project's operation. The absolute minimum details required for each camera are:

\subparagraph{Latitude and Longitude} The global position of the camera, to the highest available accuracy.
\subparagraph{Descriptive location information} City, Province, Country etc.
\subparagraph{Viewing direction} The compass direction of the camera's view.
\subparagraph{Height} Two heights should be recorded for each camera: the height above sea level of the land on which the camera is located, and the height above ground level that the camera is mounted.

\subsubsection{Camera Coverage}
With knowledge of a camera's position and height off the ground, it should be possible to establish its coverage, i.e. how much of the surrounding landscape can be seen by the camera. This will require a high resolution data set of landscape and/or building heights to establish the exact coverage of the camera, something which may or may not be freely available.

If the project can obtain such information for its cameras, it will be useful on its own to inform users of what they can see in a specific image. It will also have the additional benefit of being able to determine which cameras, if any, share the view of another camera to gain a different perspective on a specific view. This may also apply to news events, which could be defined to impact a certain area thereby allowing users to search for cameras with a view on that area.

\subsubsection{Descriptive Camera Details}
The project must provide textual details of each camera to allow users to search for cameras showing items of interest. These details will be divided into three data types:

\subparagraph{Camera description} A general description of the camera's position and views. There will be no specific requirements governing the contents of this description.
\subparagraph{Categories} A set of categories will allow camera hosts to specify the type of features seen by the camera. This will most likely be defined as a hierarchical category tree, e.g. ``Urban/City/Skyscapers'' or ``Rural/Farmland''. Each camera can be assigned multiple categories to cover the possible wide variety of visible items.
\subparagraph{Specific Items} Some cameras will inevitably be trained on specific locations of interest, e.g. the Niagara Falls. It will be possible to list these places of interest as a part of each camera's description, along with a category (e.g. Building: York Minster).

\subsection{Image Details}
The details stored with each image will be relatively simple, since most of the details of human interest will be stored in relation to the camera or the list of captured events (see below). However, some details will need to be stored with each image for the technical operation of the project, such as the format of the image file and its dimensions.

\subsection{Observed Events}
The primary purpose of the Long Webcam project is to record the events seen by its cameras. The details of these events will be stored alongside the images. The exact nature of these details will be decided at a later date, and may well change over time. However, it is anticipated that event details will include:

\subparagraph{Event dates} The date on which the event occurred. Given the long-term nature of the project, it is likely that some `events' will occur over several days.
\subparagraph{Event type} A set of categories will be created to classify events.
\subparagraph{Description} A text description of the event. This may be a short description, or a full discussion of a complex event. It may be that short and long descriptions are separated.
\subparagraph{External references} Significant events will undoubtedly be recorded elsewhere, e.g. in news broadcasts or other websites. Links to those sources will be maintained so that people can learn more about specific events from different perspectives.

\vspace{0.75cm}
The detailed description and referencing of outside materials may follow a Wiki-style structure, with descriptions and references inter-mingled. Some thought must be given to the fact that certain outside links will disappear over time. It may be possible to negotiate archiving of useful external material for the future in the event that the original source does go off-line, or to link into one of the Internet archiving projects along with the original articles.

\clearpage
\section{Project Activities}
This section describes the activities of the Long Webcam project and site. It is split into two sections: operational and user. The operational activities are those that will happen to receive, archive and maintain the data for the site. The user activities are those that visitors to the site will perform. The activities described in this section are those that will be performed purely automatically, or through interaction with the site and related systems. Person-to-person communications will not be covered unless they are related to such activities.

\subsection{Operational Activities}

\subsubsection{Notifications}
The Long Webcam server will periodically encounter situations where a human must intervene, either to check an installed camera for defects or for some other purpose. While the server will be capable of automatically sending emails to camera hosts, the possibility of false alarms will be high, especially in the early stages of the project. Consequently, any such message that the server thinks should be sent will take the form of a Notification. These notifications will be sent to the project team in the first instance, who will manually review each message and decide on an action to be taken. The message reviewer will have the following options:

\begin{itemize}
\item Send the message as is.
\item Compose a customised message (using the server message as a starting point).
\item Discard the message
\end{itemize}

As the project matures, it may be that the system becomes trustworthy enough to send messages without human intervention. A mechanism will be put in place to indicate when the server may send messages automatically. In such cases, the project team will still be made aware of the Notification for monitoring purposes.

\subsubsection{Image Ingestion}
There will be several methods through which images can be added to the Long Webcam archive. The exact method used will depend on the type of camera deployed with a given host (see Section~\ref{cameras}). This section describes how images will be retrieved; details of the storage mechanism of image files are not described.

\paragraph{Retrieval from website}
Where a camera is already posting captured images to a website, the Long Webcam site can automatically retrieve the images it needs without any input from the camera location. At midday in the camrea's timezone, the Long Webcam server will retrieve the webcam image using a supplied URL that references the webcam image. If the URL is not available, it will periodically retry the URL until an image is successfully downloaded or dusk has fallen, at which point there will be no point in retrieving an image. (The time of dusk will be calculated from standard sunrise/sunset tables which are freely available.)

When an image has been retrieved, any date/time banner on the image will be removed. In the first instance the portion to be removed will be configured in terms of a number of pixels to be removed from the top, bottom or side of the image. It may prove possible for such banners to be detected and removed automatically, in which case that approach will be used.

\paragraph{Automatic upload from a camera site}
Cameras that are attached to a standard PC, or standalone cameras with an internet connection, will be responsible for submitting images to the Long Webcam server at local midday. The Long Webcam server will listen for incoming image submissions on a standard internet port, which the software running at the camera site will be pre-configured to use. The software will capture and submit an image, together with security information to identify the specific camera host and any additional meta-data. As this information is retrieved by the Long Webcam server, it will be added to the archive.

Details of the camera software's behaviour in the event of a failure to submit images to the Long Webcam server will be described elsewhere.

\paragraph{Batch upload of images}
Cameras without an internet connection will store their captured images locally. The camera maintainer will periodically retrieve the stored images from the container and upload them to the Long Webcam site. This will be achieved through a dedicated web page on the site that the camera maintainer can access. All the stored images and their meta-data will be uploaded in a single batch, which the Long Webcam server will process and store once the upload is complete.

\subsubsection{Image schedule maintenance}
The Long Webcam site will expect to receive one image per day from each camera, with the exception of standalone cameras with no internet connection. If an image cannot be retrieved, or is not received, from a camera on a given day, it will send a Notification asking the camera host to check their camera and related equipment.

For standalone cameras, the camera host will agree to upload batches of images at certain intervals, depending on how frequently they will be able to access the camera. If a batch of images has not been received after this period, a Notification will be sent to the camera host. When a batch of images has been uploaded, the Long Webcam server will detect any missing images and alert the camera host and the project team. Although nothing can be done to replace the missing images, it will alert people to the possibility of a fault that should be investigated.

\subsubsection{Image checking}
As images are loaded into the Long Webcam archive, there may be a number of issues that can be automatically detected and acted upon by the server. If the server finds an issue with an image, it will raise a Notification requesting an appropriate action.

\paragraph{Blank image}
When a computer is unable to retrieve an image from a connected camera, it will frequently record an image of solid colour (usually grey or black), perhaps with a written caption over it. The server will be able to detect these blank images as an indication of a fault with the camera.

\paragraph{Changing image}
The server will compare each uploaded image from the previous images from that camera to see if anything major has changed. This will allow the system to detect possible changes in the viewed landscape that may be worthy of recording for future reference.

Detecting changes in this manner will be challenging, with the possibility of many false positives (e.g. foggy days). It is likely that any detected changes will be sent to the project team only in the first instance, so they can check the image archive to determine whether or not the changes are significant. If they are, the team can then send a message to the camera host requesting details of the captured event to be logged in the camera's record.

This feature may prove very useful as a determination that a camera needs cleaning!

\subsection{User Activities}
\subsubsection{Searching}
Users will typically search for specific cameras to view pictures of scenes that they are interested in. Users will be able to search for cameras by a number of criteria, either in isolation, in combination, or by filtering existing search results.

\subparagraph{Camera Location}
Users will be able to search for cameras by their location. There are several possible ways that this could be achieved:

\begin{itemize}
\item Browsing a map
\item Proximity to the user's current location (if geolocation services are available)
\item Proximity to a specified location (e.g. place name, postal code, co-ordinates)
\item Drawing an area on a map to select actual cameras, or cameras whose view encompasses that area
\end{itemize}

\subparagraph{Visible items}
Users may wish to search for cameras based on what is visible from them. The items searched for could be generic (lakes, mountains, buildings) or specific (named landmarks).

\subparagraph{Length of coverage}
If a user is looking to obtain a timelapse video from a camera's images (which is likely, since that's the major aim of the project), they will want to ensure that the chosen camera covers a sufficient timespan for their needs and/or a specific period in which they are interested.

It may be helpful for the site to calculate the timespan required for the user's video requirements (in terms of length and/or frames per second) to help specify the search criteria. This would overlap heavily with the functionality of helping users generate videos (Section~\ref{generate_video}).

\subparagraph{Searching for Times}
Users may be interested in images and videos from specific times and/or time periods. The site will provide an interface to allow searching for images from specific times.

\subparagraph{Searching for Events} Users may wish to search for images relating to specific events that cameras may have captured. Facilities will be provided to allow users to search for specific events in the cameras' histories. This search may be through keywords or a browseable timeline of recorded events.


\subsubsection{Generating videos}
\label{generate_video}
Users will be able to request a video of the view from a particular camera. The video request will be processed in a queue, and the user informed when it is ready to be downloaded. When a user requests a video, they will be able to specify the time period to be covered and the desired length (or framerate) of the video. The user may also wish to specify other video details, such as the compression rate.

The resolution of a video will be limited to some extent by the resolution of the stored images, which may change if the camera has been upgraded during the requested time period of the video. The user will be able to adjust some aspects of the video output resolution depending on the properties of the stored images.

\subsection{User support}
User support for the website will be presented in a Wiki format, covering topics for both camera hosts (in terms of recommended techniques for camera setups etc.) and to help users obtain information and videos from the site that best meet their needs.

User forums may also be introduced if it is deemed useful for site developers, camera hosts and/or users. It may work well as an initial question/answer board, whose topics could be folded into the Wiki later on.

\clearpage
\section{Use Cases}
\label{sec:use_cases}
This section describes the various activities that can be performed on the site, and an overview of what those activities will entail from the user's perspective and, in some cases, how the system will achieve them.

Each use case includes a description of which users will have access to them. A user may have a number of roles, which may vary depending on the specific context in which they are operating. These roles may be as follows:

\begin{table}[h]
\begin{tabular}{ | p{1.45in} | p{4.45in} | }
\hline
\textbf{Role} & \textbf{Description} \\
\hline
Anonymous User & A basic user who has not logged in to the site. \\
\hline
Known User & A user who has an account on the site and has logged in. Has no administrative privileges beyond their own user profile. \\
\hline
Camera Owner & A user who is responsible for maintenance of a specific camera. \\
\hline
Administrator & A user with full administration access to all areas of the site. \\
\hline
\end{tabular}
\caption{The roles that a user may hold while accessing the site.}
\label{tab:roles}
\end{table}

\section{Server Configuration}
The Long Webcam project will be hosted by a number of different servers, each of which will perform specific tasks. This division of portions of the site is a result of the computational and storage requirements of the project and the corresponding cost of servers for those activities. The different servers and their requirements are listed in Table~\ref{tab:hosts}.

\begin{table}[h]
\begin{tabular}{ | p{0.45in} | p{4in} | p{1.45in} | }
\hline
\textbf{Host name*} & \textbf{Purpose} & \textbf{Requirements} \\
\hline
\url{www} & The main web server. Provides all the functionality seen by users. Will also act as the database server for the project. & Rails/MySQL Server; Unlimited bandwidth \\
\hline
\url{images} & The storage server for all webcam images collected by the project. & Rails Server; Unlimited storage; Unlimited bandwidth \\
\hline
\url{upload} & Handles receipt of images from cameras. This will not include retrieving images from URL-based cameras. & Rails Server; Unlimited bandwidth \\
\hline
\url{proc} & Performs all video procssing and collects images from URL-based cameras. & \\
\hline
\url{beta} & Used for final testing of new features etc. before they are published on the main project site. & Separate Rails App \\
\hline
\url{dev} & Used for development and testing of new features etc. & Separate Server \\
\hline
\url{blog} & Hosts the project's blog. & WordPress and dependencies \\
\hline
\end{tabular}
\caption{Servers required to run the Long Webcam project, and a basic list of their fundamental requirements. *All host names will have the suffix \protect\url{.longwebcam.org}.}
\label{tab:hosts}
\end{table}


\subsection{\protect\url{www.longwebcam.org}}
This will be the main web server for the Long Webcam project.

\subsubsection{Functions performed}
\url{www.longwebcam.org} will be responsible for providing all the functionality to users of the site, covering the use cases described in Section~\ref{sec:use_cases}.

\subsubsection{Access}
The server will be accessible from anywhere on the web. Uploading of website files will be performed through the standard secure facilities provided by the web host.

\subsubsection{Hosting Requirements}
The design of the Long Webcam project will be such that its sites can be transferred between hosts with a minimum of effort. This places certain restrictions on the implmentation approach taken. Implementation of the site will be in Ruby on Rails with a MySQL database, due to the relative abundance of such facilities available from web hosts.

\url{www.longwebcam.org} will not have large requirements initially, but as the archive becomes larger so will its likely usage.


\subsection{\protect\url{images.longwebcam.org}}
This will be the image storage server for the project. It will be configured as a standard web server, but will not have any `normal' web pages. Instead, it will provide a number of `virtual' pages through which images can be uploaded for storage and subsequently retrieved.

\subsubsection{Functions performed}
The \url{images.longwebcam.org} server will be responsible for storing and retrieving images for the project. It will not perform any other actions. It will have access to the main site database on \url{www.longwebcam.org} to maintain the link between the database and the image archive, as well as performing appropriate security checks for uploaded images.

\subsubsection{Access}
Access to \url{images.longwebcam.org} will be restricted to requests sent by other servers in the Long Webcam project. Any requests from other locations on the Internet will be ignored.

\subsubsection{Hosting requirements}
\label{sec:images_hosting}
\url{images.longwebcam.org} will require unlimited storage due to the sheer number of images that the project intends to archive. If the site operates 200 cameras, storage requirements for new images could easily reach 500Mb per month.



\subsection{\protect\url{upload.longwebcam.org}}

This server will be responsible for handling all incoming images for the project, regardless of their source.

\subsubsection{Functions performed}

Images for the project will come from three types of camera setup, and each will be handled in different ways.

\paragraph{URL-based cameras} For cameras that upload their images to a website, the project will be given a URL for the image. The server will download an image from the URL at the local midday time in the camera's location and store it in the archive. This will almost certainly be implemented as a standalone script running periodically, outside any web server software.

\paragraph{Network-connected standalone camera} Some cameras will be connected to a PC or set up in a standalone configuration with network access. In both cases, the camera will capture an image at midday and upload it to the server. The server will receive the image and transfer it to the archive.

\paragraph{Non-networked standalone camera} Standalone camera units will store pictures captured over several days on local storage. The camera maintainer will periodically retrieve the images from the camera and upload them as a batch to the server to be added to the archive.

\subsubsection{Access}

The \url{upload.longwebcam.org} server must accept uploads from any host on the Internet. However, all attempted uploads must be authenticated against a known user of the Long Webcam project and must also supply metadata to identify the specific camera.

\subsubsection{Hosting Requirements}
\url{upload.longwebcam.org} will receive a large number of uploaded images as the number of cameras hosted by the project increases, and as such will require unlimited bandwidth. However, storage and processing requirements will be minimal.

The server must not only receive image uploads from cameras, but also have the ability to retrieve images from the web on a schedule. As such a server with full programmable and scheduling control and unlimited download bandwidth is required in addition to web server capabilities.

\subsection{\protect\url{proc.longwebcam.org}}

This server will be responsible for all the maintenance and processing tasks required by the project.

\subsubsection{Functions performed}
The most visible work done by the \url{proc.longwebcam.org} server will be the generation of timelapse videos from the Long Webcam archive as they are requested by users. However, the server will also perform routine maintenance and monitoring tasks including:

\begin{itemize}
    \item Examining new images for changes that may indicate significant events
\end{itemize}


\subsubsection{Access}
This server will not provide any web-based funtcionality, save for that required by any system monitoring tools used by the project. Any web-based access will therefore be limited to project servers only. All other access to this server will be through SSH alone.

\subsubsection{Hosting Requirements}
\url{proc.longwebcam.org} will have a number of tasks to be run at specific times that are not part of a standard web server. The fine control required for these tasks means that a standard web server will not suffice, so a server with full programming and scheduling control is required.

\subsection{\protect\url{beta.longwebcam.org}}
\url{beta.longwebcam.org} will be used to host development versions of the site to try new designs and features. Only a select few people will be invited to use this site. The site will run on the live project database and image store alongside \url{www.longwebcam.org}, and as such will have similar hosting requirements. Note that this means that any new functions added to the \url{beta} site must be thoroughly tested to ensure that the risk to the live project data is minimal.

\subsection{\protect\url{dev.longwebcam.org}}
\url{dev.longwebcam.org} will be used for the development of all project code. It will be completely independent of the main project site, with its own copies of all software, the database and the image archive. This dedicated machine will be accessible only to developers for sharing developments.

\subsection{\protect\url{blog.longwebcam.org}}
\url{blog.longwebcam.org} will host the project's blog, used to provide site news, updates etc. to interested parties. WordPress will most likely be used as the blog software.

\subsection{Choice of Hosting Solution}
Long Webcam is the sort of project that is likely to garner periodic spikes of high interest, as specific events may induce large volumes of traffic (e.g. anniversaries of major landscape-changing events - the eruption of Mount St. Helens is a good example of an event that makes frequent reappearances in the media). While this is almost certainly not a concern in the first few years of the project, it is something that must be considered to ensure that sufficient server capacity and bandwidth is available for any such occurrences as the project matures. The newest generation of cloud-based hosts are perfectly suited to handling this type of traffic as they can provide varying levels of computing power to individual sites to meet demand.

A number of web hosting firms now provide cheap plans that provide unlimited storage and unlimited bandwidth. The former option is critical to this project, given the enormous number of images that the project intends to archive. Many of these images will be provided to the project via Web interfaces, so while the needs of the \url{www.longwebcam.org} site itself may be relatively modest, the bandwidth used by the project as a whole will still be considerable.

Any webhost worth its salt will allow unlimited sub-domains to be hosted within a single hosting package. This is ideal for the Long Webcam project, which will distribute its functions amongst several URLs. Consequently, the hosting of the wide array of servers listed in Table \ref{tab:hosts} will be organised relatively simply.


\subsection{Data backups}
A web host must be chosen that keeps complete, secure and stable backups of data stored on its servers. However, this will not be fully trusted; the project will also keep copies of its data stored on local media, most likely as hard disk drives as they are cheap, small and provide large data capacity. An incremental backup strategy will need to be developed to allow prevent the need to repeatedly back up the entire archive. This strategy and any related tools will be described in Section~\ref{sec:backup}.

\subsection{Continuity of Service}
The aim for the Long Webcam project is to be fully functional for 100\% of the time, as with any other website. However, this is not possible and the project must expect some downtime in all its operations.

Certain portions of the project, such as the blog, can be considered to have relatively small consequences in case of failure, while others are more fundamental to the project's operation. Table~\ref{tab:continuity} lists the relative risk to the project of failure of each server.

\begin{table}
\begin{tabular}{ | p{0.45in} | p{0.8in} | p{4.65in} | }
\hline
\textbf{Host name} & \textbf{Risk} & \textbf{Notes} \\
\hline
\url{www} & Critical & While the web-facing front end of the project is not fundamental to the work of the project, it is nevertheless important from a professionalism standpoint that this experiences as little downtime as possible. Much more important, however, is the project's database that will also be stored on this server. Without the database, none of the other parts of the project can function. However, some resilience can added to the other components to cope in the event of a database failure, such as storing images locally until such time as they can be added to the database. The data held within the database must be multiply backed up to prevent any loss. \\
\hline
\url{images} & Critical & This server is required at all times to allow the storage and retrieval of all images captured by the project. The loss of this server and its data is equivalent to the loss of the entire project. While other project servers can be designed to handle temporary loss of access to the server and its functions, it cannot be tolerated for long. The data stored on the server must be multiply backed up to prevent any loss. \\
\hline
\url{upload} & Major & While the loss of this server cannot be tolerated for long, the remote cameras that upload images to the project will be designed to tolerate temporary loss of access to it, not only in case of a server failure but also for other connectivity problems. \\
\hline
\url{proc} & Critical & While the processing portion of this server's duties can disappear without undue risk to the project, its role as a collector of URL-based camera images (see Section~\ref{sec:camera_types}) is absolutely critical to this project. Since it is anticipated that this server will be hosted by the project team in the early stages of the project, it may well be that having a duplicate server in a physically different location is required to ensure that no images are missed. \\
\hline
\url{admin} & Major & The admin server will be used to monitor and administer all aspects of the project, and provide alerts for events that require the intervention of the project team. This role requires that the availability requirements of the server are very high. \\
\hline
\url{beta} & Low & Since this server is used only for testing purposes, it will never be critical to the running of the project. Indeed, its role implies that it is likely to be unavailable or not fully functional for periods of time. \\
\hline
\url{dev} & Major & While it is not strictly necessary for the day-to-day running of the project that this server is available, its role as a store of all the project's code means that it is vital that no data loss occurs. Therefore, all data stored on the server must be multiply backed up to  prevent any loss. \\
\hline
\url{blog} & Low & This server's role as a disseminator of non-critical information to interested parties means that it is not critical to the overall project. \\
\hline
\end{tabular}
\caption{Risks to the project of any of its servers being unavailable for any reason.}
\label{tab:continuity}
\end{table}

\section{Licensing}
The guiding philosophy of the site is that all the images stored on it will be freely avaialble for non-commercial use, providing that the Long Webcam project and the specific camera owner(s) are credited. Wherever possible the project would like any derivative works to also be available under a similar licence. To this end the default licence for all content stored by the site will be a Creative Commons Attribution-NonCommercial-ShareAlike 3.0 Unported Licence (\url{http://creativecommons.org/licenses/by-nc-sa/3.0/}). Commercial use of the site's data will not be prohibited, and neither will uses for which the default licence is not suitable for the user's purposes; in these cases, specific licences will be negotiated between the user, the project and the camera owner(s).

Some camera owners may wish to remain anonymous, and not be credited with their images. In this case any use of the images from that person's camera will only require credit to be made to the Long Webcam project.

Some camera owners may have strong reasons to have a more restrictive licence on their images. Although this status will be strongly discouraged by the project, it will accommodate such requirements if deemed necessary. The site may also allow images from these cameras to be watermarked in some way so that they cannot be used in their original form without prior permission from the camera owner.

\section{Local Backup Strategy}
\label{sec:backup}
The Long Webcam project will utilise a host that has guaranteed data backup facilities. However, this will not be trusted fully, and an offline backup of the database and all image files will be maintained. The data will be downloaded periodically from the project's servers and stored on hard drives.

While the database will probably be small enough that it can be downloaded in its entirity for each backup, the image archive is likely to be too large to accommodate this. Consequently an incremental backup strategy will be utilised, only downloading images added to the archive since the previous backup. Since the stored images will not change over time, this incremental backup will be simple to implement.

The backup will probably run on a daily schedule.


\end{document}

